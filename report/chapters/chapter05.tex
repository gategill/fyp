In this chapter is about: Pearl Pu

\section{Background}
The idea
When using K-Nearest Neighbour approaches, we only consider the neighbour of a candidate user who have already rated the item. While this does ensure that (), it also excludes users that didn't rate the movie, although they are more similar. What you end up with is with K neighbours that are ().

"The method of selecting nearest-neighbors is an important
issue for the nearest-neighbor CF approach. In [1] it is found
that highly correlated neighbors can be exceptionally more
valuable to each other in the prediction process than low correlated neighbors."

We can loosen up the condition, but then a question arises: How do we estimate a rating of an item?
Zhang and Pu \cite{pearlpu} propose a *recursive* approach to include nearest neighbours who have not rated the item in the prediction process.



1. Zhang and Pu [36] proposed a recursive algorithm that can make coarse predictions for the missing rating values of neighbouring users, thus alleviating data sparseness. The proposed approach shows promising results, achieving higher prediction accuracy than the conventional approach using PCC. ...


\section{Design}
How does it works?
Algorithm!!!
Equations!!!

\section{Implementation}
Issues or challenges when implementing it?

\section{Comparison to Previous Algorithms}